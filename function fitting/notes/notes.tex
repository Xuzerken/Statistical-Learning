\documentclass[12pt]{article}
\usepackage[UTF8, scheme = plain]{ctex}
\usepackage{amsmath}
\usepackage{amssymb}
\usepackage{amsthm}
\usepackage{mathrsfs}
\usepackage{graphicx}
\usepackage{subcaption}
\usepackage{setspace}
\usepackage{hyperref}
\usepackage{appendix}
\usepackage{listings}
\usepackage{xcolor}
\usepackage[many]{tcolorbox} %fill in tcolorbox 
\usepackage[margin=1in]{geometry}
% \usepackage[T1]{fontenc}
% \usepackage{cmbright}
\usepackage{booktabs}
\usepackage{url}
\usepackage{parskip}

\definecolor{dkgreen}{rgb}{0,0.6,0}
\definecolor{gray}{rgb}{0.5,0.5,0.5}
\definecolor{mauve}{rgb}{0.58,0,0.82}
\lstdefinestyle{myPython}{
	frame=l,
	language=Python,
	aboveskip=3mm,
	belowskip=3mm,
	showstringspaces=false,
	numbers=left,
	columns=flexible,
	numberstyle=\small\color{black},
	basicstyle={\small\ttfamily},
	keywordstyle=\color{blue},
	commentstyle=\color{dkgreen},
	stringstyle=\color{mauve},
	breaklines=true,
	breakatwhitespace=true,
	tabsize=3
}
\lstdefinestyle{myMatlab}{
	frame=l,
	language=Matlab,
	aboveskip=3mm,
	belowskip=3mm,
	showstringspaces=false,
	numbers=left,
	columns=flexible,
	numberstyle=\small\color{black},
	basicstyle={\small\ttfamily},
	keywordstyle=\color{blue},
	commentstyle=\color{dkgreen},
	stringstyle=\color{mauve},
	breaklines=true,
	breakatwhitespace=true,
	tabsize=3
}



\newtheorem{theorem}{Theorem}[section]
\newtheorem{proposition}[theorem]{Proposition}
\newtheorem{lemma}[theorem]{Lemma}
\newtheorem{corollary}[theorem]{Corollary}
\newtheorem{conjecture}[theorem]{Conjecture}
\newtheorem{definition}[theorem]{Definition}
\newtheorem{example}{Example}
\newtheorem{exercise}{Exercise}
\theoremstyle{remark}
\newtheorem*{remark}{Remark}
\newcommand{\Rule}{\rule{\linewidth}{0.5pt}}
\newcommand\real{\mathbb{R}}

\usepackage[
backend=bibtex,
style=alphabetic,
sorting=ynt
]{biblatex}
%\usepackage[backend=bibtex,style=verbose-trad2]{biblatex}
\addbibresource{ref.bib}

\title{Experiments on Using Neural Networks to Fit Function}
\author{Tao Xu}
\date{\today}

\begin{document}
	\maketitle
	\tableofcontents
	
	% \begin{abstract}
	% 	This is a report about a seires of experiments on using neural networks to fit function. 
	% \end{abstract}

	\newpage

	To simplify networks' structure, I choose to use two-layer fully-connected feedforeward neural networks to fit a 
	cosine function $cos(x)\; x\in(-\frac{\pi}{2},\frac{\pi}{2})$. 
	What I am interested is the width of network, choose of loss function and different initialization of weighted
	parameters, so I set different levels and doing experiments each for four repeatition. I experiment on $10,50,100$
	three different widths; MSE ($\sum\limits_{i=1}^{n}(f_\theta(x_i)-y_i)^2$) and MAE ($\sum\limits_{i=1}^{n}\left|f_\theta(x_i)-y_i\right|$);
	three different standard variations $0.01,0.1,1$.
	
	As for other settings, I keep them all invariant: use adam optimizer with learning rate $0.001$, 0 mean normal 
	distribution for initialization, relu $\operatorname{max}(0,x)$ as activation function, training $3000$ epoches,
	$1000$ training size, $1000$ testing size and full batch size.

	In each case, I will show the learning curve and the fitting process every 500 epoches.

	\section{Width = 10}
		\subsection{Mean Square Error}
			\subsubsection{Standard Error = 0.01}
			\paragraph{1}
			\begin{figure}[H]
				\centering  
				\subfigure[epoch=0]{
					\includegraphics[width=0.32\textwidth]{}}
				\subfigure[epoch=500]{
					\includegraphics[width=0.32\textwidth]{}}
				\subfigure[epoch=1000]{
					\includegraphics[width=0.32\textwidth]{}}
				\subfigure[epoch=1500]{
					\includegraphics[width=0.32\textwidth]{}}
				\subfigure[epoch=2000]{
					\includegraphics[width=0.32\textwidth]{}}
				\subfigure[epoch=2500]{
					\includegraphics[width=0.32\textwidth]{}}
				\caption{Curve Fitting Process}
				\label{}
			\end{figure}
			\begin{figure}[h]
				\centering
				\includegraphics[width=0.8\linewidth]{}
				\caption{Loss vs. epoch}
			\end{figure}

			\paragraph{2}
			\begin{figure}[H]
				\centering  
				\subfigure[epoch=0]{
					\includegraphics[width=0.32\textwidth]{}}
				\subfigure[epoch=500]{
					\includegraphics[width=0.32\textwidth]{}}
				\subfigure[epoch=1000]{
					\includegraphics[width=0.32\textwidth]{}}
				\subfigure[epoch=1500]{
					\includegraphics[width=0.32\textwidth]{}}
				\subfigure[epoch=2000]{
					\includegraphics[width=0.32\textwidth]{}}
				\subfigure[epoch=2500]{
					\includegraphics[width=0.32\textwidth]{}}
				\caption{Curve Fitting Process}
				\label{}
			\end{figure}
			\begin{figure}[h]
				\centering
				\includegraphics[width=0.8\linewidth]{}
				\caption{Loss vs. epoch}
			\end{figure}

			\paragraph{3}
			\begin{figure}[H]
				\centering  
				\subfigure[epoch=0]{
					\includegraphics[width=0.32\textwidth]{}}
				\subfigure[epoch=500]{
					\includegraphics[width=0.32\textwidth]{}}
				\subfigure[epoch=1000]{
					\includegraphics[width=0.32\textwidth]{}}
				\subfigure[epoch=1500]{
					\includegraphics[width=0.32\textwidth]{}}
				\subfigure[epoch=2000]{
					\includegraphics[width=0.32\textwidth]{}}
				\subfigure[epoch=2500]{
					\includegraphics[width=0.32\textwidth]{}}
				\caption{Curve Fitting Process}
				\label{}
			\end{figure}
			\begin{figure}[h]
				\centering
				\includegraphics[width=0.8\linewidth]{}
				\caption{Loss vs. epoch}
			\end{figure}

			\paragraph{4}
			\begin{figure}[H]
				\centering  
				\subfigure[epoch=0]{
					\includegraphics[width=0.32\textwidth]{}}
				\subfigure[epoch=500]{
					\includegraphics[width=0.32\textwidth]{}}
				\subfigure[epoch=1000]{
					\includegraphics[width=0.32\textwidth]{}}
				\subfigure[epoch=1500]{
					\includegraphics[width=0.32\textwidth]{}}
				\subfigure[epoch=2000]{
					\includegraphics[width=0.32\textwidth]{}}
				\subfigure[epoch=2500]{
					\includegraphics[width=0.32\textwidth]{}}
				\caption{Curve Fitting Process}
				\label{}
			\end{figure}
			\begin{figure}[h]
				\centering
				\includegraphics[width=0.8\linewidth]{}
				\caption{Loss vs. epoch}
			\end{figure}

			
			\subsubsection{Standard Error = 0.1}
			\subsubsection{Standard Error = 1}
		\subsection{Mean Absolute Error}
			\subsubsection{Standard Error = 0.01}
			\subsubsection{Standard Error = 0.1}
			\subsubsection{Standard Error = 1}

	\section{Width = 50}
		\subsection{Mean Square Error}
			\subsubsection{Standard Error = 0.01}
			\subsubsection{Standard Error = 0.1}
			\subsubsection{Standard Error = 1}
		\subsection{Mean Absolute Error}
			\subsubsection{Standard Error = 0.01}
			\subsubsection{Standard Error = 0.1}
			\subsubsection{Standard Error = 1}
	
	\section{Width = 100}
		\subsection{Mean Square Error}
			\subsubsection{Standard Error = 0.01}
			\subsubsection{Standard Error = 0.1}
			\subsubsection{Standard Error = 1}
		\subsection{Mean Absolute Error}
			\subsubsection{Standard Error = 0.01}
			\subsubsection{Standard Error = 0.1}
			\subsubsection{Standard Error = 1}
	


	
	\printbibliography
	
	\newpage

%	\begin{appendices}
%	\end{appendices}
\end{document}

